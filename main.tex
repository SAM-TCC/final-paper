\documentclass{utfpr-pg}

\usepackage{cmap}
\usepackage[T1]{fontenc}
\usepackage{graphicx}
\usepackage{latexsym}
\usepackage{amssymb}
\usepackage{lipsum}
\usepackage{pdfpages}
\usepackage{float}
\usepackage{color,soul}
\usepackage{xcolor}
\usepackage{adjustbox}
\usepackage{quoting}
\usepackage{tocloft}
\usepackage{xspace}

\usepackage[brazilian,hyperpageref]{backref}	 % Paginas com as citações na bibl
\usepackage[alf]{abntex2cite}	% Citações padrão ABNT

%\usepackage{tocloft,titletoc}

%\titlecontents{chapter}[1.8cm]{}{\bfseries\contentslabel{2.5cm}\MakeUppercase}{\hspace*{-1.5cm}}{\bfseries\titlerule*{}\contentspage}

\cftsetindents{section}{0em}{1.2cm}
\cftsetindents{subsection}{0em}{0.85cm}
\cftsetindents{subsubsection}{0cm}{0.65cm}
\cftsetindents{chapter}{0pt}{1.7cm}

\makeatletter
\renewcommand*\l@chapter{\@dottedtocline{0}{0em}{1.5cm}}

\makeatother

\usepackage[font=small,labelfont=bf]{caption}


\graphicspath{{imagens/}}

\DeclareFloatingEnvironment[
fileext=lod,
listname=Lista de Definições,
name=Definição,
placement=tbhp,
]{definicao}


\curso{ }
\autor{ Edimar Calebe Castanho  \\ Gabriel Chaves Borges }
\titulo{Sistema de Monitoramento de Apnéia}

\local{Curitiba}
\data{2020}

\orientador{Luiz Bordignon}

\preambulo{Monografia apresentada ao curso de graduação em Engenharia da Computação da Universidade Positivo, como requisito parcial para obtenção do grau de Engenheiro de Computação.\break}

% informações do PDF
\makeatletter
\hypersetup{
     pagebackref=true,
    pdftitle={\@title},
    pdfauthor={\@author},
    pdfsubject={\imprimirpreambulo},
    pdfcreator={LaTeX with abnTeX2},
    colorlinks=false,
}
\makeatother

% Controle do espaçamento entre um parágrafo e outro:
\setlength{\parskip}{0.1cm}  % tente também \onelineskip

\makeindex

\begin{document}
% Retira espaço extra obsoleto entre as frases.
\frenchspacing

\imprimircapa
\imprimirfolhaderosto



%\includepdf[pages=-]{images/termo.pdf}

%\begin{agradecimentos}
   % A todos os familiares e amigos, que estiveram presentes em todos os momentos, sendo base e apoio para o sucesso de toda a trajetória acadêmica.
    
    %A todos os professores da instituição que se disponibilizaram a nos dar suporte em diferentes ramos, além de todo o incentivo e a colaboração sobre diversos assuntos.
  
    %À Universidade Positivo, por toda a infraestrutura proporcionada e pelo incentivo constante ao nosso desenvolvimento acadêmico e profissional.
    
    %Por último e não menos importante, a Deus, por nos ter tornado quem somos hoje, por todas as suas bençãos, luz e sabedoria que nos ofereceu para que chegássemos até esse dia.
%\end{agradecimentos}
 
%  \begin{resumo}
 
 \refthis{bibli}
% O Anjo da Guarda é um sistema destinado a motociclistas voltado às atividades profissionais do dia-a-dia do entregador. Sua função é reduzir o tempo de resposta em caso de acidentes, notificando outros motociclistas próximos possibilitando a eles auxiliar o acidentado. Além de disponibilizar relatórios que indiquem as áreas mais perigosas e associação climática aos acidentes ocorridos.

% A rápida resposta aos primeiros socorros e início do tratamento especializado em caso de acidentes é um fator determinante para a sobrevivência e redução de sequelas, principalmente quanto às vítimas de acidentes de trânsito onde 35\% das fatalidades ocorrem entre 1 a 4 horas após o acidente. Apenas em 2018 houve um crescimento de 17,7\% em relação a 2017 nos acidentes de trânsito sofridos por motociclistas. Este aumento é reflexo do crescimento no setor conforme o surgimento e popularização de aplicativos de entrega.

% O sistema é composto por um aplicativo Android, sistema embarcado e servidor em nuvem. Os três combinados são responsáveis por identificar o acidente do motociclista e avisar outros usuários que estejam próximos ao local. Tanto o sistema embarcado quanto o aplicativo utilizam de giroscópio e acelerômetro para determinar a ocorrência de um acidente sofrido pelo motociclista utilizador.

%\textbf{Palavras-chaves}:  Acidente de moto, Sistema Embarcado, Acelerômetro, Giroscópio, Android
 
 %\end{resumo}
 
 
% \begin{resumo}[Abstract]
% \refthis[en]{bibli}

% The Guardian Angel is a system intended for motorcyclists focused on the day-to-day professional activities of the delivery man. Its function is to reduce the response time in case of accidents, by notifying other nearby motorcyclists enabling the injured to be assisted, as well as providing reports that indicate the most dangerous areas and climatic association with the accidents occurred.

% The rapid  first aid and the beginning of specialized accident treatment is a determining factor for survival and reduction of sequelae, especially for traffic accident victims, where 35\% of fatalities occur within 1 to 4 hours after the accident. Only in 2018 there was a growth of 17.7\% compared to 2017 in traffic accidents suffered by motorcyclists. This increase reflects the growth in the sector as the emergence and popularization of delivery apps.

% The system consists of an Android application, embedded system and cloud server, the three combined are responsible for identifying the rider's accident and warning other users who are nearby. Both the embedded system and the application use gyroscope and accelerometer to determine the occurrence of an accident suffered by the user rider.

% \textbf{Key-Words}: Motorcycle Accident,  Embedded system, Accelerometer, Gyroscope, Android
 
 
 %\end{resumo}

\pdfbookmark[0]{\listfigurename}{lof}
\listoffigures*
\clearpage
%\listoftables*

\cleardoublepage

 %\pdfbookmark[0]{\listtablename}{lot}
 %\listoftables*
 %\cleardoublepage

\pdfbookmark[0]{\contentsname}{toc}

\tableofcontents*

\cleardoublepage

\textual
\pagestyle{simple}

\captionsetup{singlelinecheck = false, justification=raggedright, labelsep=space}

\chapter{     Introdução }\label{intro}
% De acordo com o último EPISONO realizado em 2018, pesquisa realizada a cada dez anos para monitorar a qualidade do sono, 32,8\% da população de São Paulo possui apnéia do sono, uma condição clínica caracterizada por pausas respiratórias de no mínimo 10 segundos. Para que a pessoa possa sair de um episódio de apnéia, ela precisa despertar e com isso, pessoas nesta condição passam por um ciclo de sono, apnéia e despertamento que pode ocorrer de dezenas a centenas de vezes em uma mesma noite. Por este motivo, uma pessoa com essa condição tem dificuldade de entrar em sono profundo, prejudicando em muito a qualidade de seu sono. Sabemos que dormir bem é essencial para uma boa saúde e funcionamento do corpo, portanto este problema afeta consideravelmente a qualidade de vida de quem o possui [1].

% Segundo o Dr. Carlos M. Nunes, mais de 85\% dos pacientes com apneia do sono não são diagnosticados, o que significa que centenas de milhões de pessoas têm paradas respiratórias repetidas em vez de terem um sono restabelecedor e saudável a cada noite [2].

% Pensando neste cenário onde existe um problema tão relevante para a nossa sociedade com um nível tão baixo de diagnóstico, o trabalho proposto é o desenvolvimento de um hardware composto de uma braçadeira com sensor de oxigenação sanguínea, e uma máscara com sensores embutidos para a aquisição de sinais oriundos da respiração do usuário durante o sono. Esses dados são enviados remotamente para um aplicativo mobile que os envia para um servidor que por sua vez calcula se houveram pausas de respiração superiores a dez segundos e ao mesmo tempo se houve redução no nível de oxigênio no sangue, o que configura uma apneia. Todo o histórico do sono é armazenado e fica disponível para a visualização por meio do próprio aplicativo mobile ou plataforma web no formato de dashboards e gráficos. Esse fluxo pode ser observado na figura 1.

% A apneia do sono é formalmente definida como uma desordem do sono onde a respiração é interrompida repetidamente em curtos intervalos de no mínimo dez segundos. (SALMAN, 2019)

% Ainda segundo Salman (2019), existem dois tipos de apneia do sono, a apneia do sono central, onde o cérebro apresenta dificuldades em controlar a respiração durante o sono e a apneia do sono obstrutiva, onde os músculos da parte posterior da garganta obstruem as vias aéreas, sendo esta a causa majoritária de apneia do sono.

% Os dois efeitos diretos da apneia do sono é a fragmentação do sono, cujo exemplo pode ser observado na Figura 1, e redução do nível de oxigenação sanguínea que por sua vez podem causar diversos outros problemas tais como sonolência diurna excessiva, hipertensão, doenças cardiovasculares, além de problemas de memória, humor e até mesmo aumento no número de acidentes de trânsito em decorrência da sonolência ao dirigir. (KUSHIDA, 2007)

%% Contexto do problema

O descobrimento da apneia do sono pela comunidade médica é relativamente recente, tendo sua primeira menção formal em 1965 e sua primeira aparição nos livros de medicina em 1978. \cite{sono}

%% Impactos do problema

%% Nosso projeto como resposta ao problema


\section{Objetivo Geral}

Desenvolver um protótipo vestível de monitoramento do sono que realiza a detecção de apneia por meio de sensores de ar e de oximetria do sangue.
Após coletados os dados, o aparelho os envia, por meio de comunicação remota (bluetooth), para o smartphone do usuário. 
Este que, igualmente, envia os dados para um servidor que efetua o tratamento dos dados e os disponibiliza através de dashboards e gráficos em aplicativo mobile e na web, a fim de que seja realizado o correto diagnóstico e acompanhamento da condição.

\section{OBJETIVOS ESPECÍFICOS}

\begin{enumerate}
    \item Desenvolver hardware para aquisição de sinais;
    \item Desenvolver a lógica de tratamento de dados para identificar apnéias;
    \item Desenvolver backend do servidor;
    \item Desenvolver estrutura de armazenamento de dados;
    \item Desenvolver aplicativo mobile e web para a visualização dos dados de maneira gráfica;
    \item Integrar hardware de detecção com o software de tratamento de dados;
    \item Realizar testes de validação do protótipo.
\end{enumerate}



\chapter{Revisão Bibliográfica } \label{cap:rev}

\section{Fundamentação teórica}

\subsection{Apneia do sono}



\subsection{Sensor de fluxo de ar}

Existem várias tecnologias disponíveis para a medição do fluxo de ar, porém os princípios mais utilizados para este tipo de medição é por diferencial de temperatura ou por deslocamento mecânico. Os sensores baseados em diferencial de temperatura medem a queda de temperatura sofrida pela ação do fluxo de ar, alguns exemplos de implementação podem ser observados na Figura 2. Já os sensores mecânicos trabalham com ventoinhas ou comportas atreladas a potenciômetros que se movem de acordo com a intensidade do fluxo de ar. (IDA, 2020)

\begin{figure}[H]
  \caption{Exemplos de implementação de sensores de fluxo de ar por diferença térmica}
  \includegraphics[width=\textwidth]{figuras/airflow.PNG}
  \floatfoot{(a) demonstra um sensor exposto ao fluxo de ar e outro escondido, (b) demonstra dois sensores no mesmo fluxo, mas o segundo fica protegido pelo primeiro que já aqueceu o fluxo de ar anteriormente, (c) demonstra uma implementação de um sensor com quatro termístores. (IDA, 2020}
\end{figure}

\subsection{Oxímetro}

O oxímetro é um sensor óptico que mede a oxigenação sanguínea comparando a taxa de absorção de ondas eletromagnéticas de diferentes frequências. Esta relação foi descrita pela primeira vez por Johann Heinrich Lambert em 1760 e pode ser observada na Figura 3. (MOYLE, 2002)

O oxímetro emite ondas eletromagnéticas com diversas frequências dentro do espectro de aproximadamente 400 a 1000nm de comprimento de onda, estas ondas são normalmente aplicadas nas pontas dos dedos, pulso ou ouvido. As células sanguíneas atingidas absorvem este espectro de acordo com a sua taxa de oxigenação devolvendo ao sensor óptico o restante. Ao analisar este sinal que retorna ao dispositivo é possível inferir o nível de oxigenação sanguínea. (MOYLE, 2002)

\begin{figure}[H]
  \caption{Espectro de absorção de sangue oxigenado (HbO2) e deoxigenado (Hb)}
  \includegraphics[width=\textwidth]{figuras/Oximetria.PNG}
\end{figure}

\pagebreak

\section{Trabalhos correlatos}

\subsection{Um novo sistema de detecção de apneia do sono em tempo real e vestível com base no sensor de aceleração}

Neste estudo desenvolvido por Yüzer et al. (2019), foi-se desenvolvido um dispositivo portátil e barato para realizar o diagnóstico preliminar de apneia do sono, 
além de uma pulseira com a finalidade de alertar o paciente desse evento através de um motor de vibração. 
A pulseira serve como uma interrupção da apneia, para que o usuário possa voltar a respirar.

O dispositivo, que utiliza um cartão SD para armazenar os dados da apneia do sono, foi essencial para a aquisição de dados de fluxo de ar, início e fim da apneia, e o tempo total que o paciente ficou na cama. 
Este pode entregar o cartão SD para um médico, que irá analisar essas informações e diagnosticar os problemas de sono do paciente. 
Os dados armazenados no cartão podem ser interpretados através de uma aplicação em C#, também desenvolvida pelos autores do presente trabalho.

Para a identificação da apneia, os autores utilizaram um acelerômetro posicionado sobre o diafragma do paciente. 
O sistema captura as variações dos sinais desse acelerômetro para detectar os eventos de apneia ocorridos.

Com estes dados, o estudo conseguiu obter vários parâmetros a respeito da apneia do sono, tais como: 
sinal de fluxo, duração de cada apneia, posição em que o paciente está deitado, número total de ocorrências de apneia e tempo total de sono.

\subsection{Sensor sem fio baseado em magnetómetro para monitoramento da qualidade do sono}

No presente trabalho, é proposto um método alternativo para a detecção de parâmetros de sono, medindo a variação do campo magnético da Terra. 
Um sensor de magnetômetro é posicionado no peito do paciente através de uma cinta para detectar movimentos respiratórios durante o sono através das variações nos vetores magnéticos.
Esse dispositivo foi projetado para ser pequeno, para que não atrapalhe o paciente durante o sono. \cite{8252694}

Para o desenvolvimento dele, os autores utilizaram um microcontrolator ARM Cortex M0 e um módulo bluetooth low energy (BLE) nRF51822. 
O Sensor (previamente descritono parágrafo anterior) utilizado para detectar a movimentação da caixa torácica do paciente através da variação dos vetores magnéticos foi o LSM303DLHC da STMicroelectronics.
Durante uma respiração regular, o corpo move o sensor localizado no peito aproxima- damente 4mm de sua posição original. 

A média de deslocamento da caixa torácica é detectada pelo magnetômetro que mostra uma variação significativa no vetor do campo magnético, o que permite descrever a forma de onda da respiração do paciente.
O aparelho foi desenvolvido de maneira compacta, reduzindo o incômodo do paciente durante o sono, utilizando no entanto uma bateria de pelo menos 8 horas de duração em estado de medições contínuas dos parâmetros do sono.

\subsection{Monitoramento não invasivo de sinais vitais para pacientes com apneia do sono: um estudo preliminar}

Ter um sono de qualidade é necessário para a saúde mental, o bem-estar, a qualidade de vida e a segurança de uma pessoa. 
Os métodos atuais para o diagnósticos de transtornos do sono são métodos intrusivos, e podem afetar o sono do paciente. 
Como resultado, há uma necessidade crucial de sistemas menos pesados para diagnosticar problemas relacionados ao sono. Neste artigo os autores desenvolveram um sistema com vários sensores para capturar parâmetros de sono. 
Como dito anteriormente a apneia do sono é descrita como recorrentes eventos geralmente maiores que 10 segundos sem respiração. 
Estes episódios podem ser associados com desaturação de oxigênio no sangue.

No trabalho de Sadek et al. (2018), uma série de sensores foram utilizados para a detecção de eventos do sono, tais como: 
1) sensor de fluxo que coleta o ar nas vias aéreas do paciente, este preso ao seu peito através de uma cinta; 
2) sensor de oximetria de pulso nos dedos do paciente para a aquisição de oxigenação sanguínea e batimentos cardíacos por minuto; e 
3) colchão de células de carga posicionado em baixo do paciente para coletar variações da distribuição do peso durante a respiração.

Todos os dados dos sensores são enviados para um computador que então os processa, e três parâmetros podem ser extraídos dos dados crús: movimento do paciente, batimentos cardíacos e respiração. 
Os dados dos sensores foram então correlacionados e os resultados foram satisfatórios.

\subsection{ Apneia do sono, débito de sono e sonolência diurna são associados à acidentes de trânsito }

Garbarino et al. (2016) dizem que nas últimas três décadas vários estudos demonstraram uma clara relação entre sonolência excessiva diurna (SED) — ou, como os autores chamam no artigo, excessive daytime sleepiness (EDS) — e acidentes de trânsito, sendo a apneia obstrutiva do sono a principal causa médica daquela.
Infelimente, além de caminhoneiros (que são os principais tipos de pessoas que têm que realizar longas viagens em um tempo limitado) sofrerem com a apneia, fatores como obesidade, hipertensão, tabagismo, falta de exercício físico, além de idade avançada, contribuem para a atenuar.
Entre os acidentes de trânsito, foi relatado que mais da metade dos acidentes de cami- nhão causam ferimentos fatais e/ou deficiências crônicas e que o caminhoneiro é considerado culpado em mais de 80\% dos casos.

Tudo isso evidencia que o sono — mais especificamente, a falta dele — é um fator preponderante no quesito segurança tanto para quem exerce a profissão quanto para as pessoas ao redor, que estão envolvidas indiretamente (pedestres, outros veículos etc.).
Os autores coletaram dados por meios de exames médicos, entrevistas semiestruturadas e questionários. Durante o exame médico, foram registradas características básicas como sexo, idade, peso, altura e consumo de cigarro e café.

Um total de 949 motoristas de caminhão realizaram o exame clínico e preencheram os questionários; todos os sujeitos eram homens com idade média de 44,3 anos; foi constado que aproximadamente 24,3\% dos participantes tinham um quadro de débito de sono de duas ou mais horas; acidentes ocorridos nos últimos três anos foram relatados por mais de um terço dos participantes.
Uma das coisas que mais chama atenção nos resultados obtidos pelo estudo foi o do papel desempenhado pelos fatores relacionados ao sono na ocorrência de acidentes de trânsito, em detrimento de outros fatores de risco como o álcool e drogas.
% \input{Textos/Cap3.tex}
% \input{Textos/Cap4.tex}
% \input{Textos/Cap5.tex}
\clearpage
\addcontentsline{toc}{chapter}{REFÊRENCIAS}

\bibliography{biblio.bib}

\newpage

\end{document}

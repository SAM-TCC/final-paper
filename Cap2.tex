\chapter{Revisão Bibliográfica } \label{cap:rev}

\section{Fundamentação teórica}

\pagebreak

\section{Trabalhos correlatos}

\subsection{Um novo sistema de detecção de apneia do sono em tempo real e vestível com base no sensor de aceleração}

Neste estudo desenvolvido por \citeauthor{YUZER202039}, \citeyear{YUZER202039}, foi-se desenvolvido um dispositivo portátil e barato para realizar o diagnóstico preliminar de apneia do sono, 
além de uma pulseira com a finalidade de alertar o paciente desse evento através de um motor de vibração. 
A pulseira serve como uma interrupção da apneia, para que o usuário possa voltar a respirar.

O dispositivo, que utiliza um cartão SD para armazenar os dados da apneia do sono, foi essencial para a aquisição de dados de fluxo de ar, início e fim da apneia, e o tempo total que o paciente ficou na cama. 
Este pode entregar o cartão SD para um médico, que irá analisar essas informações e diagnosticar os problemas de sono do paciente. 
Os dados armazenados no cartão podem ser interpretados através de uma aplicação em C#, também desenvolvida pelos autores do presente trabalho.

Para a identificação da apneia, os autores utilizaram um acelerômetro posicionado sobre o diafragma do paciente. 
O sistema captura as variações dos sinais desse acelerômetro para detectar os eventos de apneia ocorridos.

Com estes dados, o estudo conseguiu obter vários parâmetros a respeito da apneia do sono, tais como: 
sinal de fluxo, duração de cada apneia, posição em que o paciente está deitado, número total de ocorrências de apneia e tempo total de sono.

\subsection{Sensor sem fio baseado em magnetómetro para monitoramento da qualidade do sono}

No presente trabalho, é proposto um método alternativo para a detecção de parâmetros de sono, medindo a variação do campo magnético da Terra. 
Um sensor de magnetômetro é posicionado no peito do paciente através de uma cinta para detectar movimentos respiratórios durante o sono através das variações nos vetores magnéticos.
Esse dispositivo foi projetado para ser pequeno, para que não atrapalhe o paciente durante o sono. \cite{8252694}

Para o desenvolvimento dele, os autores utilizaram um microcontrolator ARM Cortex M0 e um módulo bluetooth low energy (BLE) nRF51822. 
O Sensor (previamente descritono parágrafo anterior) utilizado para detectar a movimentação da caixa torácica do paciente através da variação dos vetores magnéticos foi o LSM303DLHC da STMicroelectronics.
Durante uma respiração regular, o corpo move o sensor localizado no peito aproxima- damente 4mm de sua posição original. 

A média de deslocamento da caixa torácica é detectada pelo magnetômetro que mostra uma variação significativa no vetor do campo magnético, o que permite descrever a forma de onda da respiração do paciente.
O aparelho foi desenvolvido de maneira compacta, reduzindo o incômodo do paciente durante o sono, utilizando no entanto uma bateria de pelo menos 8 horas de duração em estado de medições contínuas dos parâmetros do sono.

\subsection{Monitoramento não invasivo de sinais vitais para pacientes com apneia do sono: um estudo preliminar}

Ter um sono de qualidade é necessário para a saúde mental, o bem-estar, a qualidade de vida e a segurança de uma pessoa. 
Os métodos atuais para o diagnósticos de transtornos do sono são métodos intrusivos, e podem afetar o sono do paciente. 
Como resultado, há uma necessidade crucial de sistemas menos pesados para diagnosticar problemas relacionados ao sono. Neste artigo os autores desenvolveram um sistema com vários sensores para capturar parâmetros de sono. 
Como dito anteriormente a apneia do sono é descrita como recorrentes eventos geralmente maiores que 10 segundos sem respiração. 
Estes episódios podem ser associados com desaturação de oxigênio no sangue.

No trabalho de \cite{8214103}, uma série de sensores foram utilizados para a detecção de eventos do sono, tais como: 
1) sensor de fluxo que coleta o ar nas vias aéreas do paciente, este preso ao seu peito através de uma cinta; 
2) sensor de oximetria de pulso nos dedos do paciente para a aquisição de oxigenação sanguínea e batimentos cardíacos por minuto; e 
3) colchão de células de carga posicionado em baixo do paciente para coletar variações da distribuição do peso durante a respiração.

Todos os dados dos sensores são enviados para um computador que então os processa, e três parâmetros podem ser extraídos dos dados crús: movimento do paciente, batimentos cardíacos e respiração. 
Os dados dos sensores foram então correlacionados e os resultados foram satisfatórios.

\subsection{ Apneia do sono, débito de sono e sonolência diurna são associados à acidentes de trânsito }

\cite{101371} dizem que nas últimas três décadas vários estudos demonstraram uma clara relação entre sonolência excessiva diurna (SED) — ou, como os autores chamam no artigo, excessive daytime sleepiness (EDS) — e acidentes de trânsito, sendo a apneia obstrutiva do sono a principal causa médica daquela.
Infelimente, além de caminhoneiros (que são os principais tipos de pessoas que têm que realizar longas viagens em um tempo limitado) sofrerem com a apneia, fatores como obesidade, hipertensão, tabagismo, falta de exercício físico, além de idade avançada, contribuem para a atenuar.
Entre os acidentes de trânsito, foi relatado que mais da metade dos acidentes de cami- nhão causam ferimentos fatais e/ou deficiências crônicas e que o caminhoneiro é considerado culpado em mais de 80\% dos casos.

Tudo isso evidencia que o sono — mais especificamente, a falta dele — é um fator preponderante no quesito segurança tanto para quem exerce a profissão quanto para as pessoas ao redor, que estão envolvidas indiretamente (pedestres, outros veículos etc.).
Os autores coletaram dados por meios de exames médicos, entrevistas semiestruturadas e questionários. Durante o exame médico, foram registradas características básicas como sexo, idade, peso, altura e consumo de cigarro e café.

Um total de 949 motoristas de caminhão realizaram o exame clínico e preencheram os questionários; todos os sujeitos eram homens com idade média de 44,3 anos; foi constado que aproximadamente 24,3\% dos participantes tinham um quadro de débito de sono de duas ou mais horas; acidentes ocorridos nos últimos três anos foram relatados por mais de um terço dos participantes.
Uma das coisas que mais chama atenção nos resultados obtidos pelo estudo foi o do papel desempenhado pelos fatores relacionados ao sono na ocorrência de acidentes de trânsito, em detrimento de outros fatores de risco como o álcool e drogas.